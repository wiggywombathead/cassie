\section{Implementation}

\label{sec:implementation}

In this section we shall detail our implementation of the peer prediction
mechanism. The system consists of five independent parts, each responsible for
a separate area of functionality. These are as follows:

\begin{enumerate}
	\item Database
	\item Trading
	\item Arbitration
	\item Server
	\item User experience
\end{enumerate}

\subsection{Database}

\subsubsection{Table Definitions}

The database is responsible for persistent storage. It is implemented using the
Mito library provided by Quicklisp, which is an object relational mapper that
provides support for MySQL, PostgreSQL, and SQLite3. This allows us to define
and interact with the database tables while remaining within the Lisp
ecosystem, leading to a quicker and more flexible development cycle.

We define three tables that handle all interactions with the database:
\code{USER}, which stores all users in the system and their remaining budget;
\code{SECURITY}, which is stores the wager, deadline, number of shares, and
final outcome for every user-created security; and \code{USER-SECURITY}, which
is responsible for the many-to-many relationship that users may have with
securities, and allows us to store user positions as well as reports once the
deadline has passed. The table definitions are as follows:

\begin{table}[ht]
	\centering
	\begin{tabular}{|c|c|c|}
		\hline
		\textbf{User} & \emph{name} & \emph{budget} \\ \hline
	\end{tabular} \\~\\

	\begin{tabular}{|c|c|c|c|c|}
		\hline
		\textbf{Security} & \emph{bet} & \emph{shares} & \emph{deadline} &
		\emph{outcome} \\ \hline
	\end{tabular} \\~\\

	\begin{tabular}{|c|c|c|c|c|c|c|}
		\hline
		\textbf{User-security} & \emph{user} & \emph{security} & \emph{shares}
		& \emph{report} & \emph{positive belief} & \emph{negative belief} \\
		\hline
	\end{tabular} \\~\\
\end{table}

An entry in the \code{USER} table consists of a username and a budget. All
users currently start with \$100, which is of no real consequence since the
market uses play-money. Currently there is no requirement to supply a password
when logging; this has been done to speed up the testing process, and if
required in the future this will be a simple addition to make.

Securities are also simple entities to store, whose fields fully describe a
market: \emph{bet} stores a string that specifies the wager being made;
\emph{shares} stores the total number of shares (referred to as $q$ in
Section~\ref{sec:marketScoringRules}); \emph{deadline} holds the date and time
by which the event's outcome will have been realised; and \emph{outcome} is
initially null and eventually set to the payout price per share, or the
fraction of arbiters reporting a positive outcome, once the deadline has
passed.

The \code{USER-SECURITY} table is used to represent the mapping between users
and securities, and is used to store a user's position in a given market as
well as the outcome they have reported on its outcome, if they are indeed an
arbiter for it. The columns are as follows: \emph{user} holds a reference to an
entry in the \code{USER} table; \emph{security} holds a reference to an entry
in the \code{SECURITY} table; \emph{shares} stores the user's position in this
market if they have one, otherwise 0; \emph{report} stores the user's report on
the outcome, if they are an arbiter for this security; and \emph{positive
belief} and \emph{negative belief} represent how reliable a user's signals are,
based on their previous reporting history. The manner in which we use the last
two fields will be discussed in greater depth in Section~\ref{sec:arbitration}.

\subsubsection{Database Interface}

We provide the interface with which we interact with the database in
\code{database.lisp}. We first initialise the database by connecting to our
choice of backend and then define the tables as in the previous section. We opt
to use MySQL as the backend simply as it was already installed on the
development machine, as well as some prior familiarity with the language.
Tables are then created using the \code{deftable} macro supplied by Mito:
syntactically, this is similar to vanilla Common Lisp's \code{defstruct} macro.
Listing~\ref{lst:deftable} shows how we define the columns and their associated
datatypes. The macro defines the default accessors\footnote{Functions for
accessing members of a struct.}, the slots \code{created\_at} and
\code{updated\_at}, and a primary key \code{id} if none is specified. As the
listing shows, we can specify a previously defined table as the column
datatype, in this case \code{user} and \code{security}, in order to model the
foreign key relation in a straightforward manner. 

\begin{lstlisting}[
	label={lst:deftable},
	caption={Defining the \code{USER-SECURITY} table in Mito}]
(deftable user-security ()
          ((user :col-type user)
           (security :col-type security)
           (shares :col-type :integer
                   :initform 0)
           (report :col-type (or :integer :null))
           (positive-belief :col-type (or :double :null))
           (negative-belief :col-type (or :double :null))))
\end{lstlisting}

Insertion is similarly straightforward: to insert a new entry into a table we
simply create an instance of the structure that is implicitly defined when calling
\code{deftable} then call \code{create-dao}. To retrieve records from the
database we can use either \code{select-dao} or \code{find-dao}: the former
returns all records satisfying the criteria provided, while the latter returns
the first match.

We design the interface to the database so that no custom queries need to be
created outside of \code{database.lisp}. This ensures the code interacting with
it can be kept as clean and simple as possible. We use another library from
Quicklisp, SXQL, in order to build the more complex queries to the database.
For example, Listing~\ref{lst:retrieveExpiring} shows how we retrieve all the
securities whose deadline is yet to pass, in order of first to last to expire.
We use the SXQL functions \code{where}, \code{:>}, \code{order-by}, and
\code{:asc} that expand into the corresponding MySQL code for Mito to execute.

\begin{lstlisting}[
	label={lst:retrieveExpiring},
	caption={Retrieving active markets using Mito and SXQL}]
(defun get-expiring-securities (datetime)
  " return all securities ordered by deadline, most imminent first "
  (with-open-database
    (select-dao 'security
                (where (:> :deadline datetime))
                (order-by (:asc :deadline)))))
\end{lstlisting}

For any interaction with the database, we need to establish a connection prior
to the transaction and disconnect after it is complete. This gives us the
opportunity to make use of Lisp's macro system: while it is a small example,
since its use is so widespread it drastically reduces the number of lines and
ensures we never forget to close a database connection. We define our macro
\code{with-open-database} as in Listing~\ref{lst:databaseMacro}. Since the
final statement in a function or macro definition in Lisp is the value returned
by that block, we are able to open the connection, execute arbitrary code and
store the final result in the variable \code{result}, then disconnect from the
database and return the result of the transaction.

\begin{lstlisting}[
	label={lst:databaseMacro},
	caption={Defining our \code{with-open-database} macro}]
(defmacro with-open-database (&body code)
  " execute CODE without worrying about the connection "
  `(progn
     (connect-database)
     (let ((result (progn ,@code)))
       (disconnect-database)
       result)))
\end{lstlisting}

In order to enable the transactions in the following sections, when we first
initialise the database we create a user with the name ``bank''. All money that
is then to be collected from or paid out to users is done so through the bank,
so that money is largely conserved. Although not so important for a play-money
market, it does help to enforce a value on the currency we use as well as
analyse the system's performance.

\subsection{Trading}

Trading allows us to gather public sentiment on the user-defined securities,
and this part of the system is responsible for setting share prices,
calculating the cost of transactions, and charging fees to raise funds for the
arbitration stage, to ensure arbiters are incentivised to report truthfully.
These features are implemented in the files \code{msr.lisp} and
\code{market.lisp}.

As discussed in Section~\ref{sec:marketScoringRules}, we will be using a Market
Scoring Rule as our automated market maker. In order to achieve the guarantees
of Freeman, Lahaie, and Pennock's mechanism we must use a scoring rule that is
strictly proper, which can then be implemented as a market maker based on a
convex cost function. We use the commonly-used Logarithmic Market Scoring Rule
(LMSR) created by Robin Hanson~\cite{Hanson2007}, whose cost function is
defined as follows:

\begin{equation}
	\label{eq:LMSR}
	C(\vect{q}) = b \log \left( \sum_j e^{q_j/b} \right)
\end{equation}

This assumes that each security $j$ is one of a collection of mutually
exclusive and exhaustive outcomes. Since we are only dealing with binary
events, we can compute a share price based only on the number of shares bought
for the positive outcome, and assume that buying shares in the negative outcome
is equivalent to selling shares in the positive outcome. In this case, we have
$\vect{q}=( 0 \quad q_1 - q_0 )$, where $q_0$ and $q_1$ are the quantity of
shares bought by agents in the negative and positive outcomes, respectively.
This gives us the following cost function for LMSR in the binary setting, where
$q=q_1-q_0$:

\begin{equation}
	\label{eq:LMSRbinary}
	C_b (q) = b \log (1 + e^{q/b})
\end{equation}

In these cost functions, $b$ appears as a parameter that allows us to control
the responsiveness of $C$. A lower value of $b$ corresponds to a more sensitive
share price, meaning the price will change more quickly for smaller
transactions. It also controls the market's risk of loss: for markets with
$|\Omega|$ outcomes it can be shown that the maximum loss incurred by the
market maker is $b \log |\Omega|$. Recall that to compute the actual cost to
charge an agent for a transaction, we compute $C_b(q')-C_b(q)$ for an agent
wishing to take the total quantity of shares from $q$ to $q'$, and this also
encodes sell transactions. The share price function is the derivative of the
cost function, which is in our case:

\begin{equation}
	\label{eq:LMSRprice}
	p_b(q) = \frac{e^{q/b}}{1+e^{q/b}}
\end{equation}

Upon market creation a user is only given the option to buy a positive number
of shares -- otherwise, it would make more sense for the user to create a
market for the opposite outcome. After this, the custom security is created and
is open for all other users to trade in. The share price reflects the strength
of the community's opinion on the event's outcome: for example, at $q=0$ then
the community is exactly split on whether the event will have a positive
outcome, and appropriately $p_b(0)=\$0.50$ for any $b$. Meanwhile, for $q=20$
and $b=10$ this means twenty more shares have been bought than sold in the
market and would yield a quoted share price of $p_{10}(20) \approx \$0.88$.
Users feel more confident that the event will be positive, thus pushing the
price upwards. This could assure some agent that the event will have a positive
outcome. Suppose they wish to buy ten more shares in the market: they would
then be required to pay $C_{10}(30)-C_{10}(20)=10 \log (\frac{1+e^3}{1+e^2})
\approx \$9.22$. Such an action in the market would push the share price to
$p(30) \approx \$0.95$. Users may also be required to pay a fee on their
transaction, in order to subsidise the arbitration stage.
Figure~\ref{fig:marketCreation} shows the interface for creating a new market,
while Figure~\ref{fig:transactionSummary} shows an example of the transaction
summary that a user is presented with after having done so.

\begin{figure}[h]
	\centering
	\includegraphics[width=\textwidth]{market-creation}
	\caption{The interface for creating a new market}
	\label{fig:marketCreation}
\end{figure}

\begin{figure}[h]
	\centering
	\includegraphics[width=\textwidth]{transaction-summary}
	\caption{Users are presented with a transaction summary upon creating a new
	market}
	\label{fig:transactionSummary}
\end{figure}

Fees are only charged on transactions where an agent increases their risk: this
means they are either buying or selling a greater number of shares than their
current position. For example, a user owning ten shares and selling five would
incur no extra cost since they are liquidating a position they have already
invested in, while selling more than ten or buying additional shares would
incur an additional cost. The fee serves a secondary purpose in bounding the
price of the security away from \$0 and \$1, as a potential trader would be
spending more than \$1 or less than \$0 to buy or sell the shares. For example,
even if an agent were to buy an infinitesimal number of shares, meaning their
transaction cost would be given by the share price function $p$, if the share
price was \$0.99 and the transaction fee was set to 5\%, they would be required
to pay $\$0.99 \cdot 1.05 = 1.0395$, greater than the maximum possible payout.

\subsection{Arbitration}

\label{sec:arbitration}

\subsubsection{Computing signal reliability}

The arbitration stage is where we resolve market outcomes using arbiter reports
and pay out winnings to, or demand payment from, stakeholders in the security.
Since the mechanism incentivises arbiters to act truthfully even if an arbiter
themselves holds a stake in the market, we decide to let anyone opt in to
reporting on the outcome. Once a market's deadline has passed the security will
be listed as an expired market and the user is presented with the option to act
as an arbiter, as in Figure~\ref{fig:expiredMarket}. After doing so, they may
then input their observed signal, or indeed a lie, as
Figure~\ref{fig:reportOutcome} shows.  Once the required number of reports have
been collected, we move onto rewarding arbiters for submitting reports and
computing the final outcome of the market.

\begin{figure}[h]
	\centering
	\includegraphics[width=\textwidth]{expired-market}
	\caption{Users may opt in to report on market outcomes}
	\label{fig:expiredMarket}
\end{figure}

\begin{figure}[h]
	\centering
	\includegraphics[width=\textwidth]{report-outcome}
	\caption{The interface by which arbiters report market outcomes}
	\label{fig:reportOutcome}
\end{figure}

Arbiters are rewarded by being paid a certain amount of money only if their
report agrees with another randomly chosen arbiter. The reward is determined by
the 1/prior-with-midpoint mechanism, where instead of using the common
prior probability $\mu$ we use the update probabilities $\mu_1$ and $\mu_0$.
Recall that these are the probability that, given that arbiter $i$ receives a
positive signal so too does another randomly chosen arbiter $j$, and the
probability that, given that $i$ receives a negative signal, another randomly
chosen arbiter $j$ receives a positive signal. Since we cannot assume arbiters
to act truthfully, this is more complicated than simply counting the types of
reports received. Instead, we use information about signal error rates
$Pr[x_i=1|X=1]$ and $Pr[x_i=1|X=0]$ based on past market outcomes and past
reporting behaviour for each arbiter $i$. This is a fairly involved process:
first we gather all securities that have been reported on in the past by $i$
and whose outcome has been determined; we then iterate through each one of
these returned securities and determine the report that $i$ submitted as well
as its outcome, and push this into a list of results. At the end of this stage
we have a list \code{((security report outcome) ...)} that describes the
arbiter's report and the market's true (peer-determined) outcome for each
security for which they have acted as an arbiter. The code implementing this is
show in Listing~\ref{lst:reportingHistory}.

\begin{lstlisting}[
	label={lst:reportingHistory},
	caption={Gathering a user's reporting history}]
(defun get-securities-reported-by-user (user)
  (with-open-database
    (select-dao 'security (inner-join 'user-security
                                      :on (:= :security.id
                                              :user-security.security-id))
                (where (:and (:= :user-security.user-id (user-id user))
                             (:not-null :security.outcome)
                             (:not-null :user-security.report))))))

(defun get-reporting-history (user)
  (let ((securities (get-securities-reported-by-user user))
        security-reports)
	(dolist (security securities)
	  (with-open-database
	    (let ((report (user-security-report
	                    (find-dao 'user-security
	                              :user user
	                              :security security)))
              (outcome (security-outcome security)))
	      (if outcome
            (push (list security report outcome) security-reports)))))
	security-reports))
\end{lstlisting}

We then use this history to compute the probability that, assuming the arbiter
has always acted truthfully, their signal has been correct in telling them the
true outcome of the event. We refer to these probabilities, $Pr[x_i=1|X=1]$ and
$Pr[x_i=1|X=0]$, as a arbiter $i$'s positive and negative signal beliefs as a
remnant of a previous implement. Listing~\ref{lst:positiveBelief} shows how we
calculate an arbiter's positive belief, with a similar method applying for
computing the negative belief: for the arbiter's history restricted to the
securities whose outcome was reported as positive by the majority of arbiters,
we count the number of times this arbiter also reported a positive outcome,
thus giving us the arbiter's signal error probability given we know the outcome
is (more likely to have been) positive. If there have been no positive
outcomes, then we assume the arbiter has a perfect signal. Calculating the
negative signal belief is done in a similar manner: first we collect items of
the arbiter's reporting history where the security was mostly reported as a
negative outcome, then we count the number of times this arbiter submitted a
positive report. Again we assume in the lack of markets with negative outcomes
that the arbiter has a perfect signal. We repeat this to compute the positive
and negative signal beliefs for each arbiter.

\begin{lstlisting}[
	label={lst:positiveBelief},
	caption={Computing an arbiter's positive signal belief given
	their reporting history}]
(defun calculate-positive-belief (reporting-history)
  ;; first only get the reports where the outcome was positive
  (let ((positive-outcomes (remove-if-not #'(lambda (x) (>= 0.5 (third x)))
                                          reporting-history))
        positive-reports)

    ;; now count the number of times the user reported a positive outcome
    (setf positive-reports (count T (mapcar #'(lambda (x) (equal 1 (second x)))
                                            positive-outcomes)))
    (if positive-outcomes
      (/ positive-reports (length positive-outcomes))
      1)))
\end{lstlisting}

We use this information about the reliability of each arbiter's signal, along
with the prior probability $\mu$ that the event had a positive outcome, to
compute the update probabilities $\mu_1$ and $\mu_0$, used in the
1/prior-with-midpoint mechanism. We use these to compute the update
probabilities $\mu_1^i$ for each arbiter $i$, using a randomly chosen peer
arbiter $j$, as follows:

\begin{equation}
	\label{eq:updateProbabilities}
	\begin{aligned}
		\mu_1^i & = Pr[x_j=1|x_i=1] \\
		& = Pr[x_j=1|X=0] \cdot Pr[X=0|x_i=1] + Pr[x_j=1|X=1] \cdot Pr[X=1|x_i=1]
	\end{aligned}
\end{equation}

We use the same approach to compute $\mu_0^i$ for each $i$. The final values of
$\mu_1$ and $\mu_0$ are then calculated by taking the minimum and maximum
across all $\mu_1^i$ and $\mu_0^i$, respectively.

\subsubsection{Rewarding the arbiters}

We may now pair arbiters randomly to pay them via the 1/prior-with-midpoint
mechanism. We first retrieve two lists from the database: the first is the list
of all arbiter reports and is of the form \code{arbiter-reports = ((arbiter
report) ...)}; the second is a list of all arbiter signal beliefs and is of the
form \code{arbiter-beliefs = ((arbiter positive-belief negative-belief) ...)}.
Since we will be pairing arbiters randomly and still need efficient access to
these values, we then create two hash tables associating an arbiter to their
report and their beliefs, giving us \code{report[i] = report} and
\code{beliefs[i] = (positive-belief negative-belief)} for each $i$. The
market's outcome is simply set to the proportion of arbiters that reported a
positive outcome. At this point we pay out the appropriate winnings to
stakeholders, paying out money to those who hold shares and demanding money
from those who have gone short. The former group are looking to be paid out at
a price higher than the level at which they bought the shares, while the latter
are hoping to buy back their shares at a lower level than their ``in'' price.
Now that the market's outcome has been determined it will no longer be listed
as an unresolved market on the user's dashboard.

\begin{lstlisting}[
	label={lst:marketOutcome},
	caption={Computing the market outcome}]
(let ((reports (mapcar #'second arbiter-reports)))
  ;; the payoff of each share held is the fraction of arbiters reporting 1
  (setf outcome (float (/ (count 1 reports) (length reports)))))
\end{lstlisting}

We next compute the random pairing of arbiters. Lisp has no function to shuffle
a list, so we implement the Fisher-Yates~\cite[pg.~26-27]{FisherYates1938}
algorithm ourselves as the \code{shuffle} function in
Listing~\ref{lst:randomPairing}. The random pairing is then formed by walking
through the shuffled list and collecting adjacent elements as pairs.

\begin{lstlisting}[
	label={lst:randomPairing},
	caption={Assigning arbiters to peers randomly}]
(defun shuffle (lst)
  " shuffle LST randomly (without modifying it) "
  (let ((lst (copy-list lst)))
    (loop for i from (length lst) downto 2 do
          (rotatef (elt lst (random i))
                   (elt lst (1- i))))
    lst))

(defun random-pairing (lst)
  " randomly pair items from LST together if length(lst) is even "
  (if (evenp (length lst))
    (loop for (a b) on (shuffle lst) by #'cddr while b
          collect (list a b))))
\end{lstlisting}

For each set of paired arbiters we then retrieve their reports and signal
beliefs from the hash tables and compute, for each arbiter $i$ in the pair, the
values of $\mu_1^i$ and $\mu_0^i$ according to
Equation~\ref{eq:updateProbabilities}, where the randomly chosen $j$ is simply
the partner with whom they have been paired. We push these values to a list so
we may then compute the minimum value of $\mu_1^i$ and maximum value of
$\mu_0^i$ across all arbiters. Finally to compute the smallest $k$ to satisfy
Equation~\ref{eq:kParameter} we set $k = \max_i 2|n_i|/m\delta$, and use this
to pay arbiters according to Equation~\ref{eq:oneOverPrior}.

\subsection{Server}

The code from each of the separate areas of the market is then drawn together
in the file \code{server.lisp}, in which we set up the web server, define the
webpages, and call the functions from the different interfaces we provide.

We use Hunchentoot to host the web server. At startup this involves creating an
instance of a Hunchentoot \code{easy-acceptor}, which opens up a port of our
choosing to accept requests. Doing so also initialises the dispatch table,
which is a list containing the functions to execute when the corresponding
webpage is loaded. We can specify webpages to the dispatch table by using
Hunchentoot's \code{create-prefix-dispatcher} function: this simply takes a URL
and the function to execute when that URL is loaded. Here is again the
opportunity to make use of Lisp's macro system: the code in
Listing~\ref{lst:defineURL} shows a macro that defines a function and a URL of
the same name, creates a dispatcher from them and pushes it to the dispatch
table. This again not only saves rewriting repetitive code but also keeps the
codebase clear and logical -- the URL \code{/index} is served by a function
called \code{index}. Now each time we wish to define a new webpage we need only
to call \code{define-url-fn} followed by the name of the page and the
instructions to execute.

\begin{lstlisting}[
	label={lst:defineURL},
	caption={Macroising URL functions}]
(defmacro define-url-fn ((name) &body body)
  " creates handler NAME and pushes it to *DISPATCH-TABLE* "
  `(progn
     ;; define the handler
     (defun ,name ()
       ,@body)

     ;; add the handler to the dispatch table
     (push (create-prefix-dispatcher
             ,(format NIL "/~(~A~)" name) ',name)
           *dispatch-table*)))
\end{lstlisting}

To define the content of the webpages we use CL-WHO\footnote{Common Lisp With
HTML Output.}, translates Lisp statements into strings of valid HTML. Since we
are aiming for a uniform style across all webpages -- for example, with a
consistent navigation bar, a content section, and all the same preamble -- this
is another opportunity to make use of Lisp's macro system, to avoid code
duplication and improve the abstraction to the code. In this vein we define the
\code{standard-page} macro, an abridged version of which is given in
Listing~\ref{lst:standardPage}. This allows us to concisely define webpages
with a similar look and feel in a concise manner, and only requires us to
specify the content that makes the page unique via the \code{body} parameter.

\begin{lstlisting}[
	label={lst:defineURL},
	caption={Macroising webpage definitions}]
(defmacro standard-page ((&key title) &body body)
  " template for a standard webpage "
  `(with-html-output-to-string
     (*standard-output* NIL :prologue T :indent T)
     (:html :xml\:lang "en"
            :lang "en
            (:head (:title title)
                   (:link :href "/style.css" :type "text/css" :rel "stylesheet")
                   ;; rest of preamble ... )
            (:body
              (:ul :id "navbar"
                   (:li ... ))
              (:div :class "container"
                    ...
                    (:div :class "content"
                          ,@body))))))
\end{lstlisting}

Another feature of Hunchentoot ... 

% session handling
% sending data e.g. (parameter ...)

\subsection{User Experience}

% AJAX and Javascript

%Defining such tables using Mito is as simple as using the \code{deftable}
%macro, and is syntactically similar to defining a regular struct within Common
%Lisp. Listing~\ref{lst:deftable} shows how we define the columns and their
%associated datatypes. The macro defines default accessors\footnote{Functions
%for accessing members of a struct.}, the slots \code{created\_at},
%\code{updated\_at}, and a primary key \code{id} if none is specified. Insertion
%is similarly straightforward: to insert a new entry into a table we create an
%instance of the class that is implicitly defined when calling \code{deftable},
%then call \code{create-dao}. To retrieve records from the database we use
%either \code{select-dao}, which returns all records matching the conditions, or
%\code{find-dao}, which returns only the first.
%
%\begin{lstlisting}[
%	label={lst:deftable},
%	language=lisp,
%	captionpos=b,
%	caption={Defining a table in Mito}]
%(deftable security ()
%    ((bet :col-type :text)
%     (shares :initform 0
%             :col-type :integer)
%     (deadline :col-type :datetime)
%     (outcome :col-type (or :double :null))))
%\end{lstlisting}
%
%\subsection{Trading}
%
%Trading allows users to create arbitrary bets on binary events and then trade
%shares in them. It also implements the trading fees on risk transactions
%required to raise funds to reward arbiters and pay out stakeholders. To
%calculate the trading fee, we first need to know whether the user is increasing
%their risk (a ``risk transaction'') or they are simply liquidating a position
%they already hold, and secondly whether they are buying or selling shares,
%since these incur different charges.
%
%To create markets, quote their share prices, and compute the cost of a
%transaction we use a market scoring rule. We use the commonly used Logarithmic
%Market Scoring Rule (LMSR), created by Robin Hanson~\cite{LMSR}. As we have
%mentioned, in any MSR an agent wishing to purchase $q'-q$ shares of a given
%security must pay $C(q')-C(q)$ for convex, differentiable, monotonically
%increasing function $C$. In the case of LMSR, for current number of shares $q$
%and liquidity parameter $b$, this function is:
%
%\begin{equation}
%	\label{eq:LMSR}
%	C_b(q) = b \log (1 + e^{q/b})
%\end{equation}
%
%This function also accommodates selling shares, in which case $q'<q$.  Since a
%user must specify a desired amount of shares to buy or sell before we can
%determine how much to charge them, we cannot use this function to directly
%quote a share price. Instead, we can calculate the cost of the transaction were
%the user to purchase a miniscule quantity of shares and quote this as the share
%price. This is simply the derivative of Equation~\ref{eq:LMSR} and is given by:
%
%\begin{equation}
%	\label{eq:LMSRprice}
%	p_b(q) = \frac{ e^{q/b} }{ 1 + e^{q/b} }
%\end{equation}
%
%Although Equation~\ref{eq:LMSRprice} quotes the share price, it is not accurate
%to use it to calculate how much an agent should pay, since the agent's
%participation in the market will immediately change this price. It does,
%however, provide a useful means of describing to the user the perceived
%likelihood of the event. For example for some security, at $q=0$ an equal
%number of shares have been bought as have been sold, meaning the userbase as a
%whole is equally unsure about whether the market will have a positive or
%negative outcome. The quoted share price will be $p(0)=\$0.50$.  Similarly,
%suppose $b=10$ and $q=20$, meaning twenty more shares have been bought than
%sold in the market. This would yield a share price of $p_{10}(20) \approx
%\$0.88$. Users feel more confident that the event will have a positive outcome,
%since more people have bought shares, giving a higher share price. As mentioned
%in Section~\ref{sec:oneOverPrior}, we interpret the closing price of the market
%as the probability of a positive outcome, as it approximates how sure users
%feel about the event's outcome.
%
%\begin{figure}[h]
%	\centering
%	\includegraphics[width=.8\textwidth]{market-creation}
%	\caption{Creating a market for the wager, ``The weather will be hot on
%	Friday''}
%	\label{fig:marketCreation}
%\end{figure}
%
%\subsection{Arbitration}
%
%\label{sec:arbitration}
%
%The arbitration stage of the mechanism seeks to resolve markets whose deadlines
%have passed by gathering reports from arbiters and paying out, or demanding
%money from, stakeholders. Since we will eventually be able to accommodate for
%users to act as arbiters in market in which they themselves hold a position, we
%allow any user to opt in to become an arbiter in an unresolved market.
%Figure~\ref{fig:dashboard} shows the dashboard as it appears to a logged in
%user, in which they are presented with all the currently active and unresolved
%markets. 
%
%\begin{figure}[h]
%	\centering
%	\includegraphics[width=.8\textwidth]{dashboard}
%	\caption{A user's dashboard, where they can see all active and unresolved
%	markets}
%	\label{fig:dashboard}
%\end{figure}
%
%Once the user has opted in to arbitration they are taken to a form to submit
%their report on the outcome, as well as estimations on their ``Positive
%Belief'' and ``Negative Belief''. These are estimations of probabilities that,
%given that the event \emph{actually} had a positive (respectively negative)
%outcome, the user received a positive signal. This is important as we need to
%take into account signal noise, given that wagers can be subjective: while the
%positive and negative signal beliefs will be nearer to 1 and 0, respectively,
%for an event such as the winner of a football match, since it will likely be
%covered by many media outlets all reporting the true result, this may not
%necessarily be the case where the outcome is more a matter of opinion. The
%weakness of this approach is discussed in Section~\ref{sec:evaluation}.
%
%\begin{figure}[h]
%	\centering
%	\includegraphics[width=.8\textwidth]{arbitration}
%	\caption{The interface for submitting a report as an arbiter}
%	\label{fig:resolveSecurity}
%\end{figure}
%
%These signal beliefs are used to compute the update probabilities $\mu_1$ and
%$\mu_0$ which are used in the 1/prior-with-midpoint mechanism. Given the
%closing price $\mu$ and the positive and negative signal beliefs
%$Pr[x_i=1|X=1]$ and $Pr[x_i=1|X=0]$ for each arbiter $i$, we can compute the
%positive update $\mu_1^i$ for each $i$ and randomly chosen $j$ as follows:
%
%\begin{equation}
%	\begin{aligned}
%		\mu_1^i & = Pr[x_j=1|x_i=1] \\
%		& = Pr[x_j=1|X=0] \cdot Pr[X=0|x_i=1] + Pr[x_j=1|X=1] \cdot Pr[X=1|x_i=1]
%	\end{aligned}
%\end{equation}
%
%We use the same approach to compute the negative update $\mu_0^i$ for each $i$,
%and hence we may now also compute the common updates $\mu_1$ and $\mu_0$ to pay
%arbiters the correct reward for arbitration. Once done, we set the outcome of
%the market to the fraction of arbiters that reported a positive outcome, paying
%users who hold shares and taking money from users that sold shares short.
%
%\subsection{Server}
%
%The code contained in \code{server.lisp} is responsible both for setting up and
%maintaining the Hunchentoot web server and defining the various pages and forms
%with which users interact.  Hunchentoot makes interaction with the web server
%straightforward. We initialise the server by creating an \code{easy-acceptor},
%and once we have defined our webpages using CL-WHO we simply push them to the
%dispatch table using \code{create-prefix-dispatcher} so that they may be
%accessed. Hunchentoot also provides us with automatic session handling, meaning
%we do not have to worry about the details of logging in users on different
%machines at the same time: we simply define the symbol \code{session-user} in
%the appropriate session-handling data structure, then set this to the logged in
%user. We can then access this using \code{(session-value 'session-user)} to
%display different material on the page and interact with the database according
%to the current user. Finally, we use Hunchentoot to access the GET and POST
%parameters users send to the server via forms with a call to the library's
%\code{parameter} function. 
%
%Although there is little special to CL-WHO compared to other HTML-generating
%libraries, it is useful to remain in the Lisp environment to define webpages
%since we may use its powerful macro system. We use it to define a template for
%a standard page, meaning we only need to specify the elements that make the
%page unique and giving the web application a consistent style. This also
%enables us to define pages and add the corresponding HTML generating function
%to the dispatch table all within a single interface, hiding the unnecessary
%details. Listing~\ref{lst:defineURL} shows how we are able to define both the
%webpage, generate its content, and push it to the dispatch table in one.
%
%The code in \code{server.lisp} draws together the interfaces from other areas
%of the system and actually executes the mechanism. This involves taking user
%input to create custom markets, quote share prices, and collect and distribute
%payments for the various interactions users may have with the market.
%
%\begin{lstlisting}[
%	label={lst:defineURL},
%	language=lisp,
%	captionpos=b,
%	caption={Defining webpages}]
%(defmacro define-url-fn ((name) &body body)
%  `(progn
%     (defun ,name () ,@body)
%     (push (create-prefix-dispatcher ,(format NIL "/~(~a~)" name) ',name)
%           *dispatch-table*)))
%\end{lstlisting}
