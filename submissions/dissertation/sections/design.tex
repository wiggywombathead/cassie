\section{Design}

% different types of market:
% - market scoring rule
% - call market auction
% - continuous double auction
% - bookmaker
% - parimutuel market

% preselecting users
% number of users to report?

\label{sec:design}

\subsection{Mechanism Overview}

As mentioned our design of the prediction market is based on the peer
prediction mechanism proposed by Freeman et al.~\cite{Freeman2017}. In this
section we will present the main ideas presented in their work and give an
overview of the mechanism itself.

We are interested in setting up a prediction market for outcome of a binary
random variable $X \in \{0,1\}$. We will use the terms ``market'', ``stock'',
and ``security'' interchangeably throughout to refer to the entity comprising a
wager, such as ``Arsenal will beat Tottenham'', and a deadline -- these two
pieces of information are all we need to represent the event $X$. The mechanism
is divided into two main stages: the market stage, where users may buy and sell
shares in the securities whose deadlines have not yet passed; and the
arbitration stage, where a subset of the users report on the outcome of the
security and the payout price per share is computed. In a traditional
prediction market on binary events, if the market's outcome was positive then
stakeholders with long positions will then be paid out \$1 for each share they
own while those with short positions will buy back their shares at a price of
\$1 per share. Similarly, if the outcome was negative, long users will have
lost money since they receive no money back from their initial investment,
while short users will profit as they must ``buy back'' their shares at a price
of \$0 per share. This market is different in that market outcomes are set as
the proportion of users that reported a positive outcome. Therefore, even if a
user has gone long on a security, this must have been at or below the right
price to make a profit, since it is not necessarily the case that 100\% of the
users reporting on the market agree on its outcome.

Since we rely heavily on user participation for the mechanism to run correctly,
it is important that users act in the desired manner. This mechanism
incentivises users to act truthfully in two aspects: firstly, users are
encouraged to trade on their belief of the market's realised outcome, rather
than, for example, how a specific news source will report on it, since this
outcome is determined entirely by the community, each individual of which will
have access to their own news sources of varying biases; secondly, it is in a
user's best interests to report on market outcomes truthfully, since they can
receive no better payoff by attempting to manipulate the system. The same is
true even if an otherwise rogue reporter held a position in the market.
Therefore, we are able to gather accurate public sentiment on the event itself
as well as its outcome, and ambiguous securities are dealt with more gracefully.

These considerations also allow us to achieve certain useful guarantees. For
example, in order to incentivise reporters to act truthfully, we must pay them
more than what they would otherwise gain from attempting to manipulate the
system. We can use this knowledge, coupled with what we know stakeholders are
expecting to be paid out, in order to bound the amount we must pay to
ensure incentive compatibility. In this way we can ensure that the system is
sustainable and the system's loss is never unbounded.

In the following sections, we shall outline the mechanism we implement from a
theoretical standpoint. 

\subsection{Market Stage}

The market stage allows users to create markets for any bet they desire and
specifies the how the share price reacts according to user participation. As we
implement a decentralised market, we place no restriction on the bet that can
be placed (only that it is binary, i.e. it is ultimately either a ``yes'' or a
``no''). In any market, there are two sides to any trade: one side offers some
number shares, while there is some party that must be willing to purchase this
number of shares at this price. There are a number of options for implementing
such a trading mechanism, as we will detail in the following section.

\subsubsection{Market Scoring Rules}

\label{sec:marketScoringRules}

There are a  number of options for implementing a market trading mechanism:
these include as a continuous double auction, market call auction, or by using
an automated market maker. A continuous double auction (CDA) is a method by
which buyers are matched with sellers of a stock. The market maker keeps an
order book that tracks the bids, submitted by those looking to buy a stock, and
the asks, submitted by those looking to sell a stock. Traders arrive
asynchronously and place orders, and when two opposite orders match the trade
is executed. Continuous double auctions are traditionally used in highly
liquid markets, where there are many bids for a given ask (and vice versa),
such as the New York Stock Exchange. An issue with this approach, however, is
that it relies on high liquidity -- that is, it requires that there is always a
willing buyer and seller for a particular price at a particular quantity of
shares. Especially in a prediction market, most prediction markets have far
fewer participants than a stock exchange: if Alice is willing to sell a share
of stock $A$ for \$6 but no one is willing to buy at that price, then a trade
cannot be executed. This problem can be particularly severe in a combinatorial
market, where a trader's attention is split between exponentially many
securities. Traders may also be hesitant to participate in a market organised
as a continuous double auction, since their participation at a particular price
encourages others to participate at a lower price thus depriving them of
profit. Prices may therefore not be informative, and instead only reflect the
game traders are playing with one another.

To avoid these problems of illiquidity, one can use an automated market maker,
in which a price maker is (nearly) always willing to accept both buy and sell
orders at a certain price. It is possible that this price changes with the
user's interaction with the market. This ensures that participants are always
able to make a trade, thus ``making'' the market. Typically this is not used in
real-money markets since always assuming the opposite side to any given trade
would likely result in losses for the house; in play-money markets this is not
such an issue as losses are less detrimental, but prices must still be adjusted
to ensure losses are bounded. Two options arise for implementing an automated
market maker: Market Scoring Rules (MSR) and parimutuel markets. We are
interested in the former.

Suppose our market contains $|\Omega|$ mutually exclusive and exhaustive
securities. Let $q_j$ be the total quantity of shares held by all traders of
security $j$, and let $\vect{q} = (q_1, \ldots, q_{|\Omega|})$. At the heart of
a Market Scoring Rule is the cost function $C(\vect{q})$, which is a means of
recording the total amount of money spent by traders as a function of the total
number of shares in circulation. An agent wishing to purchase $q_j' - q_j$
shares of security $j$, increasing the number of shares of security $j$ to
$q_j'$, would then pay $C(\vect{q}_{-j}, q_j') - C(\vect{q})$.\footnote{We use
$\vect{q}_{-j}$ to denote the vector $(q_1, \ldots, q_{j-1}, q_{j+1}, \ldots,
q_{|\Omega|})$.} This model also accommodates selling shares, in which case
$q_j' < q_j$. We cannot use $C$ to quote a share price to the user, since we
first need to know the quantity of shares they wish to buy or sell: instead, we
use its derivative $p = \partial C / \partial q_j$, which gives the cost for
purchasing an infinitesimal quantity of shares. Just as we cannot use $C$ to
quote individual share price, we cannot use $p$ to calculate the cost of
transaction, since a user's interaction with the market will automatically have
an effect on share price as, in practice, they are not purchasing an
infinitesimal quantity.

\subsubsection{Trading fees}

In addition to implementing a Market Scoring Rule to quote the share price of a
given security and calculate the necessary payments for buying or selling
shares in a given market, the market stage is responsible for implementing
trading fees. As mentioned, the mechanism we implement is not budget balanced,
meaning the market must be subsidised in order to pay the correct winnings to
stakeholders when market outcomes are realised. Trading fees are therefore used
to raise these subsidies. For any given market, buying shares will push the
share price $p$ upwards towards \$1, while selling shares will push it towards
\$0. There are two types of transactions that a user may be involved in: one in
which a trader is increasing their risk, and one in which a user is liquidating
shares it has previously bought or sold. For example, suppose Alice holds ten
shares in a particular security: if she were to sell up to and including ten
shares she would simply be liquidating shares that have already been sold to
her, while if she were to buy additional shares or sell more than ten, then she
would be increasing her risk. Risk transactions are defined analogously for a
user buying shares. Trading fees are only imposed on transactions in which a
user increases their risk, and can be viewed as a fee on the worst-case loss
incurred by the agent. Specifically, for fixed system parameter $f$ and for
transactions in which a user increases their risk, a buy transaction that
pushes share price to $p$ incurs an additional charge of $fp$, while a sell
transaction that pushes share price to $p$ incurs additional charge of
$f(1-p)$. Users may trade shares in a given security as long as they have
enough funds to make the transaction (including the fee), and as long as the
deadline has not yet passed. After the market expires, stakeholders' positions
are final and we then determine the outcome of the market via peer prediction
in the arbitration stage.

\subsection{Arbitration Stage}

The arbitration stage is concerned with determining the perceived outcome
of the event $X$ from a subset of the total userbase, known as the
``arbiters'', who offer reports on what they observed the outcome to be.
Specifically, each arbiter $i$ receives a private signal $x_i \in \{0,1\}$ that
tells them the result of the event -- this is analogous to reading the news,
watching the match, even hearing about it from a friend, and will vary from
market to market. The arbiter then submits a report $\hat{x}_i \in \{0,1\}$ to
the system that tells it what they believe the outcome to be. Note that since
the signal they receive is private information, we have no way of determining
whether this report is what they truly observed, or whether they are trying to
manipulate the system for their own gain. Instead, we incentivise arbiters to
act truthfully by paying them a reward if their report agrees with another
randomly chosen arbiter: for this we implement the ``1/prior with midpoint''
mechanism, which we will detail below.

Once all reports have been collected and the arbiters paid, the outcome of the
market $\hat{X} \in [0,1]$ is set to the proportion of arbiters that reported a
positive outcome. This differentiates this mechanism from traditional
prediction markets, in which shares of a security will pay out \$1 if the event
occurred, and \$0 otherwise. Stakeholders are then paid out in the usual, where
those with long positions are paid out $\hat{X}$ for each share owned, while
those with short positions must buy them back at $\hat{X}$ per share. This
should not change how traders view the security: if they have information
telling them the event will occur they will still buy into the market if the
share price is appropriate, while if they believe the event is unlikely they
will continue to sell. The mechanism simply accommodates for the ambiguous bets
possible as a result of being made by the community.

\subsubsection{1/prior mechanism}

We use a modified version of the 1/prior payment mechanism to reward users for
submitting reports on an event's outcome and to incentivise truth telling
behaviour. The original version was conceived by Jurca and
Faltings~\cite{JurcaFaltings2008, JurcaFaltings2011} as a means of rewarding
arbiters for participation in opinion polls, another means of crowdsourcing a
forecast in which users submit probabilistic estimates for the likelihood of
events to occur. Witkowski~\cite{Witkowski2014} then generalised this to pay
out different amounts depending on the signals reported by paired arbiters. For
arbiters $i$ and $j$ with reports $\hat{x}_i$ and $\hat{x}_j$, the 1/prior
mechanism pays a reward $u(\hat{x}_i, \hat{x}_j)$ as follows:

\begin{equation}
	\label{eq:oneOverPrior}
	u(\hat{x}_i, \hat{x}_j) =
	\begin{cases}
		k \mu & \text{if } \hat{x}_i = \hat{x}_j = 0 \\
		k (1-\mu) & \text{if } \hat{x}_i = \hat{x}_j = 1 \\
		0 & \text{otherwise}
	\end{cases}
\end{equation}

In this, $k$ is a parameter and $\mu$ is the common prior belief that $X=1$. A
suitable value to use for $\mu$ in our case is the closing price of the market:
if users feel the event is likely to occur they will buy shares of it, pushing
the share price towards \$1, and if they feel it is unlikely it will be pushed
towards \$0. Everyone can see this price, and if it differs from a user's
beliefs they will participate in the market and alter the price accordingly,
making it a sensible choice for the common prior.

The modification introduced by Freeman et al.~\cite{Freeman2017} of the 1/prior
mechanism is simple and requires two extra values to be computed. Let $\mu_1^i$
be the probability that, given that agent $i$ receives a positive signal of the
event's outcome, another randomly chosen user also receives a positive signal.
Similarly, let $\mu_0^i$ be the probability that, given that agent $i$ receives
a negative signal, another randomly chosen user receives a positive signal. We
require a common value for these ``update'' probabilities across all agents, so
we define the $\mu_1$ and $\mu_0$ as follows:

\begin{equation}
	\begin{gathered}
		\mu_1 := \min_i \mu_1^i \\
		\mu_0 := \max_i \mu_0^i
	\end{gathered}
\end{equation}

The modified payment mechanism is now simply Equation~\ref{eq:oneOverPrior}
with $\mu$ replaced by $(\mu_1 + \mu_0)/2$. This is the
``1/prior-with-midpoint'' mechanism and guarantees that the incentives for
arbiters are always the same, no matter the signal they receive. The arbiter
with the greatest incentive to misreport -- that is, an arbiter with a large
stake in the market in which they are reporting -- has their incentive to
misreport weakly decreased by using the midpoint $(\mu_1 + \mu_0)/2$ in the
payment rule as opposed using $\mu$ in the standard version.

In particular, suppose arbiter $i$ holds a position of $n_i$ securities in the
market. We can ensure truthful reporting is a best response for $i$ by setting
the 1/prior-with-midpoint parameter $k$ to the appropriate value such that they
will receive weakly greater reward from the payment mechanism than they would
by misreporting. With $m$ arbiters and $\delta = \mu_1 - \mu_0$, truthful
reporting for arbiter $i$ is a best response if:

\begin{equation}
	\label{eq:kParameter}
	k \ge \frac{2 |n_i|}{m \delta}
\end{equation}

\subsection{Tools}

% TODO

%At the minimum we require the ability to host a web server, define webpages,
%and interact with a database management system. We have chosen to implement the
%project almost entirely in Common Lisp, for which there is a variety of
%libraries offering such capabilities. Using only Common Lisp, as opposed to a
%combination of, for example, HTML, PHP, and MySQL, allows us to remain in a
%single uniform environment and make use of Lisp's powerful macro system and
%functional style of programming. We also use Git and Github for version
%control.
%
%The market has been developed and tested with the Steel Bank Common Lisp (SBCL)
%compiler and runtime environment. From here we can use Quicklisp, a library
%manager that provides 
%
%
%The project is implemented in Common Lisp and the code has been developed and
%tested within the Steel Bank Common Lisp (SBCL) compiler and runtime system.
%Code version control has been achieved with Git and Github.  Writing the web
%application has required the use of several libraries available from the
%library manager Quicklisp, specifically:
%
%\begin{itemize}
%	\itemsep0em
%	\item Hunchentoot
%	\item CL-WHO
%	\item Mito
%	\item SXQL
%	\item Parenscript
%	\item Smackjack
%\end{itemize}
%
%Hunchentoot provides the environment on which we host the server. Most
%importantly it provides automatic session handling, allowing for multiple users
%to be logged in at once, and easy access of GET and POST parameters, enabling
%interaction via HTML forms. To generate the webpages, we use CL-WHO, which
%converts Lisp statements into strings of valid HTML. Defining webpages while
%remaining in the Lisp environment means we may use Lisp's macro system to build
%abstractions for both defining structure and processing data in one interface.
%
%Mito and SXQL provide the ability to connect to and interact with a Relational
%Database Management System (RDBMS). We use MySQL, though this choice is largely
%immaterial given our simple requirements of the database.
%
%Parenscript incorporates Javascript into the site with the goal of improving
%user experience. Currently it is only used to ensure that all necessary fields
%during market creation, trading, and market resolution are non-empty to avoid
%sending incomplete data to the server. It will be used to a greater degree in
%the future to ensure responsiveness: all information displayed to the user must
%be current to ensure that users are interacting with an up-to-date state of the
%system. We therefore plan to make more use of Parenscript and Smackjack, an
%AJAX library for Lisp, to allow for asynchronous interaction with the server.
%This will include, for example, continuously updating a stock's price or
%calculating the cost of a transaction without a page refresh, and stronger
%client-side validation.
%
