\section{Literature Review}

% Iowa Electronic Markets
% PredictIt, PredictAlot
% Intrade
% Rational expectations - demand not same as traditional market
% PredictIt disputes: wording

\subsection{The Iowa Electronic Markets}

The Iowa Electronic Markets (IEM) are real-money prediction markets developed
by the University of Iowa~\cite{IEM} that have been running since 1988. They
allow users to buy and sell contracts based on the outcome of U.S. political
elections and economic indicators, and are currently offering markets for the
winning party of the 2020 U.S. presidential election, the vote share between
the Democratic and Republican parties in the 2020 U.S. presidential election,
and the compositions of the houses of Congress, House of Representatives, and
U.S. Senate after the outcome of the 2020 U.S. congressional elections. The
number of markets offered is small and the topics are kept relevant to current
events, meaning there is likely to be high liquidity for any security a user
wishes to trade in. This has allowed the markets to predict the results of
political elections with more accuracy and less error than traditional polls:
for the presidential elections between 1988 and 2000, three-quarters of the
time the IEM's market price on the day each poll was released was more accurate
for predicting vote share than the poll itself~\cite[pg.~19]{WisdomOfCrowds}.
These markets inspired similar markets in the forms of the Hollywood Stock
Exchange, NewsFutures, and the Foresight Exchange report, which achieved
similar successes despite not using real money. 

\subsection{Combinatorial Markets}

An issue with these markets is that they are restrictive in the bets they
offer. Although this can be beneficial in that they provide a focused and
liquid market in which to trade, it leaves them potentially less interesting to
interact with. A combinatorial prediction market is a solution to this and
drastically increases the number of events that can be bet on and outcomes
predicted by offering securities on interrelated propositions that can be
combined in various ways. One example of such a market is that of
\emph{Predictalot}~\cite{Predictalot}, a combinatorial prediction market
developed by Yahoo! that allowed users to trade securities in the 2010 NCAA
Men's Division I Basketball Tournament. The tournament sees the top 64 teams
play 63 games in a knockout competition, yielding a total outcome space of size
$2^{63}$. \emph{Predictalot} then kept track of the odds, computing them by
scanning through all of the predictions made by users. This prediction market
was the original inspiration for this project in investigating prediction
markets. Using a Market Scoring Rule for such a market would involve computing
a summation over the entire outcome space $\Omega$, an intractable, \#P-hard
problem akin to counting the number of variable assignments that satisfy a CNF
formula, or or the number of subsets in a list of integers that sum to zero.
Instead, they employ an implementation of importance sampling, a technique for
estimating properties of a particular probability distribution using only
samples generated from a different distribution. This ``naive'' approach is
then improved upon by the work of Dud\'ik, Lahaie, and
Pennock~\cite{Dudik2012}, who use convex optimisation and constraint generation
to develop a market maker which is tractable. This approach lies somewhere
between treating all securities as independent and a fully combinatorial,
``ideal'' market maker, propagating information among related securities for
more complete information aggregation. The ways in which their odds are
calculated are also natural: for example, a large bet on a team to win the
entire tournament automatically increases the odds that the same team will
progress past the first round, since they would not be able to win the
competition without doing so. This work is then improved upon by Kroer,
Dud\'ik, Lahaie, and Balakrishnan~\cite{Kroer2016}, in which they use integer
programming to achieve arbitrage-free trades, or always profitable risk-free
trades. On top of achieving bounded loss, a crucial element behind a market
mechanism operating in the real world and avoiding bankruptcy, avoiding
arbitrage is desirable as it leads to more accurate forecasts: since users
cannot make risk-free profits, they are forced to bet according to their true
beliefs.

\subsection{Decentralised Markets}

All examples so far have involved a centralised market mechanism. These types
of systems involve a central authority providing the bets upon which users may
bet and then verifying their outcome. \emph{Decentralised} markets allow the
users themselves to define their own bets and trade shares in them. These types
of systems involve a central authority providing the bets upon which users may
bet and then verifying their outcome. These types of markets allow the users
themselves to define their own markets by providing custom bets and them
trading shares in them. Several examples of decentralised markets exist and
they are often implemented with cryptocurrencies. Peterson et
al.~\cite{Peterson2015} study the setting and use it to implement the oracle at
the heart of \emph{Augur}~\cite{Augur}, a decentralised prediction market built
upon the Ethereum blockchain that launched in 2018. It allows users to offer
predictions on any topic, and markets may be either categorical, which are
similar to binary markets in which the winner takes all, or scalar, which offer
users a spectrum of outcomes in which to invest. For example, users may bet
that the global average temperature for 2020 will lie in a certain range.  As
in many decentralised markets, outcomes of events are then resolved by the
users, and in the case of \emph{Augur} users are incentivised to report
truthfully by way of paying reporting fees. This amounts to users depositing
tokens to back their report, and token holders are then entitled to the trading
fees generated. Although this persuades against manipulation, it has not been
shown whether this system achieves any theoretical guarantees of truthful
reporting.

As can often be the case with real-money markets, the platform had quickly
devolved into an assassination market~\cite{AugurDeathMarket} -- originally
this referred to the case where users created markets on the deaths of certain
people, which then incentivised their assassination. A user could stand to
profit by placing a bet on the exact time of their death, and ensure this bet
was profitable by assassinating the subject. More generally this refers to the
users of a prediction market having the ability to influence a market's outcome
and acting on this opportunity. Another issue with \emph{Augur} is the option
to report a market's outcome as ``invalid'': this is for the case where the
user-made bet is too ambiguous to decided, such as, ``Bayern Munich will play
well against Paris Saint Germain''.

Other decentralised markets based on cryptocurrencies exist, including
\emph{Omen}~\cite{Omen} and \emph{Hivemind}~\cite{Hivemind}. The former is
similar to \emph{Augur} in that it allows users to create markets for any bet
they like and whose outcomes are not decided by the system itself. Whereas
\emph{Augur} uses a reputation system in the form of requiring users to back
their report of a market's outcome with \$REP tokens, \emph{Omen} asks the
market creator to supply an ``oracle'' through which the outcome can be
determined. They note that this oracle can even be \emph{Augur}. Although this
may solve the ``invalid'' outcome option for ambiguous bets, it may introduce
bias into the process of outcome determination. For example, suppose a user
creates a market for ``The Democratic nominee will tell a lie during tonight's
debate'' and lists the oracle as the conservative news channel, Fox News. Users
would then trade on how they think the oracle will report the outcome, and not
what they believe the outcome will be themselves. An important aspect of
decentralised markets must therefore be that the outcome is determined by the
community, not a single source. 

In contrast to \emph{Augur}, which implements a traditional orderbook using
Ethereum, \emph{Omen} uses an automated market maker to provide liquidity to
its securities. Two approaches to this include using a Market Scoring Rule to
update the odds on a given event, and implementing a parimutuel market where
users compete for a share of the total money wagered while the share price
varies dynamically according to some cost function. Hanson~\cite{Hanson2003}
shows that we can use any strictly proper scoring rule to implement an
automated market maker: with such scoring rules, agents maximise their expected
utility by truthfully revealing their predictions. In particular, in this
project we implement the peer prediction market introduced by Freeman, Lahaie,
and Pennock~\cite{Freeman2017}, which specifies a market that uses a Market
Scoring Rule to trade bets and crowdsources outcome determination, similarly to
\emph{Augur}, by asking users for reports. All they require is that the rule is
strictly proper, giving plenty of choice to study the effects different rules
have on user behaviour. While the choice of scoring rule is less important than
the mechanism by which market outcomes are determined, a recent work by Liu,
Wang, and Chen~\cite{Liu2020} introduces scoring rules for the setting where
the aggregator has access only to user reports, which they call Surrogate
Scoring Rules (SSRs). This appears to be an interesting avenue to further
explore and adapt to a prediction market. One assumption they make, however,
seems incompatible with the decentralised setting in that they require all
events to be independent.  Given that users can create a market for \emph{any}
bet, this condition is impossible to ensure. Since SSRs can be strictly proper
under certain conditions, they may be applicable as the Market Scoring Rule in
\cite{Freeman2017}. 

Other prediction markets existed in \emph{PredictIt}~\cite{PredictIt} and
\emph{InTrade}~\cite{InTrade}, both offering markets for various political and
economic events. However, both experienced disputes largely related to the
wording of the available to trade and the ambiguity in their resolution. For
example, a bet offered on \emph{PredictIt} was, ``Who will be Senate-confirmed
Secretary of State on March 31, 2018?'', and although Rex Tillerson was fired
in the middle of March, he was officially the secretary of state until midnight
of the 31st, leading to confusion among users in what they are trading on and
therefore inaccuracies in the predictions it elicited. This will be a problem
inherent to any prediction market that allows users to specify their own bets,
and while the work of Freeman et al.~\cite{Freeman2017} evades the problem of
determining the outcome by setting it to the proportion of users reporting a
``yes'' outcome, this does nothing to penalise the initial creator for
introducing an ambiguous market.

\begin{figure}[h]
	\centering
	\includegraphics[width=\textwidth]{predictit}
	\caption{The \emph{PredictIt} prediction market for the 2020 U.S.
	presidential election. As of \today\ Joe Biden is perceived to be more
	likely to become President.}
	\label{fig:predictit}
\end{figure}
