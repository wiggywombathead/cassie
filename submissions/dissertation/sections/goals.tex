\section{Goals}

\label{sec:goals}

\subsection{Core Features}

The goal of this project is to implement a truthful decentralised prediction
market in which users may specify the bets on which to trade. The market
outcomes will be decided by peer prediction, in which events are settled by a
subset of the users known as arbiters. A user may even act as an arbiter in a
market in which they themselves may hold a stake. Specifically, we seek to:

\begin{itemize}
	\item create a web application on which to host the prediction market
	\item allow users to create custom bets
	\item implement a trading mechanism that allows users to buy and sell
		shares in the user-made securities
	\item crowdsource outcome determination using reports from arbiters who may
		hold positions in the market
	\item incentivise truthful behaviour at all stages in the mechanism
\end{itemize}

These goals will largely be achieved by implementing the mechanism outlined by
Freeman et al.~\cite{Freeman2017}, albeit wit several practical modifications.
The first three goals cover core functionality of any decentralised prediction
market, while the fourth is concerned with the tuning of system parameters,
before and during execution, in order to ensure that users do not manipulate
the mechanism. In this users are not only discouraged from attempting to
``game'' the system for personal gain, they are hurt for doing so.

Other non-essential features but highly desirable for strong user experience
include asynchronous communication with the server in order to display
up-to-date pricing information to the user without a page refresh, and the
automated closing of markets. Both of these features would make the system
straightforward and intuitive to use. Moreover, it would allow the system to
run independently, meaning the market's functioning is only influenced by the
community, one of the key points of implementing a decentralised market.

As we will discuss in more detail in Section~\ref{sec:design}, one aspect of
the mechanism is the assumption that the system knows the signal error rates
when user's receive news about a market's outcome. This is unrealistic in
practice, since we cannot hope to know the exact where each user learns about
the outcome of the event nor the accuracy of said source's reporting. Hence we
also look to implement a way in which users do not need to be explicitly asked
their estimates of signal accuracy, and instead this is calculated based on
their past reporting history. This leaves less opportunity to game the system,
the entire point of implementing this prediction market mechanism.

\subsection{Stretch Features}

With more time, there are plenty of additional features that could be
implemented to render the system more intuitive and usable. These include the
option to create different types of markets, particularly categorical ones
since they would function similarly to binary markets but allow multiple
related markets to be expressed more succinctly. Furthermore the option to
create and sort markets by categories would help users offer their information
more readily if they are especially interested in a certain topic, say politics
or sport.

A useful feature to implement would be the tracking of price histories for each
security. This would enable graphs to be generated so that users could be more
informed on how the forecast of an event has changed over time. This would make
the decision to participate at a particular price point more interesting on
their part since they are not only estimating their belief on the probability
that the event will have a positive outcome, but also on how the other users
will act, similar to a real stock market.

Finally, an issue with the mechanism of Freeman et al. as it stands is that it
does not directly punish users for creating markets on ambiguous bets. Traders
may become confused about the wording, leading to different interpretations and
hence different users trading on different beliefs -- this is not useful for
information aggregation. Although the negative effects are somewhat mitigated
in that the very same community that trades in the security also decides on its
outcome, there is no mechanism in place to specifically encourage clear bets.
This is something that could be improved and would be useful in avoiding the
market becoming swamped with overly subjective wagers.

\subsection{Motivation}

As discussed, there are already numerous decentralised markets that exist in
the literature.
