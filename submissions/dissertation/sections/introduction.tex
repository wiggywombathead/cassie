\section{Introduction}

\label{sec:introduction}

Prediction markets are exchange-traded markets\footnote{Markets in which all
transactions are routed through a central source.} which allow users to trade
on the outcomes of future events as opposed to traditional financial
instruments. Users participate by placing bets and buying or selling shares in
the markets these bets give rise to. Since users stake their own money, market
prices should indicate the true beliefs of the userbase and the perceived
likelihood the events have of occurring. Different users will have different
beliefs and knowledge informing their decisions, and prediction markets provide
a means of aggregating information on events of interest using ``the wisdom of
the crowd''. Shares in these markets are usually traded between \$0 and \$1,
and for binary events a market will typically pay out \$1 for every share held
for a positive outcome, and \$0 otherwise.

Traditional prediction markets are centralised in the sense that the system
provides the bets for the users to trade in, and traders simply choose the
price point and size of stake at which to participate. Since the system knows
all possible bets beforehand it is simple to allow it to determine the outcomes
of the markets and pay out the winnings to stakeholders accordingly. There are
two issues with this centralised approach: firstly, it restricts the types of
bets that can be made, since they must be explicitly offered by the market
maker; secondly, it operates on trust -- there is nothing to stop the central
market maker from manipulating the system for their own gain. We aim for a
system that avoids both of these issues.

In this project we implement a \emph{decentralised} prediction market in which
it is up to the userbase itself to define the markets and determine the
outcomes of these events -- these outcomes are decided upon by consensus among
a group of users, known as arbiters. This removes the need for a trusted
centre, however with no central moderator bets may become ambiguous or their
outcomes subjective, and arbiters may still attempt to manipulate the outcome
of the market for their own gain by submitting false reports. It is also
important that users continue to act according to their true beliefs so we may
learn about the true public sentiment on the events. We base our design on the
incentive compatible outcome determination mechanism proposed by Freeman,
Lahaie, and Pennock~\cite{Freeman2017}, which allows us to crowdsource market
outcomes while incentivising users to act truthfully in all stages of the
prediction market.

The rest of this dissertation is structured as follows: in
Section~\ref{sec:background} we introduce the various types of prediction
markets and different approaches by which they can be implemented. In
Section~\ref{sec:literatureReview} we give an overview of existing
implementations and the current literature within the algorithmic game theory
community. In Section~\ref{sec:goals} we state our goals for the project and
justify why we think this project is worthwhile. Section~\ref{sec:design}
covers the high-level design of the market on which we base our implementation
and introduces some theoretical considerations to take into account. In
Section~\ref{sec:implementation} we discuss the details of our implementation
of the decentralised market, including the tools we use to do so.
Section~\ref{sec:projectManagement} covers our approach towards the management
of the project. Finally, in Section~\ref{sec:evaluation} we reflect on the
project's successes and suggest areas for improvement.
